\begin{tabular}{lrrrrrrrr}
\toprule
 & \multicolumn{2}{r}{icl_accuracy} & \multicolumn{2}{r}{tv_accuracy} & \multicolumn{2}{r}{icl_comet} & \multicolumn{2}{r}{tv_comet} \\
 & mean & std & mean & std & mean & std & mean & std \\
model &  &  &  &  &  &  &  &  \\
\midrule
llama_7B & 0.520000 & 0.457200 & 0.456000 & 0.417200 & 0.851900 & 0.121700 & 0.752600 & 0.212700 \\
pythia_2.8B & 0.748000 & 0.130800 & 0.384000 & 0.265900 & 0.923600 & 0.026800 & 0.832800 & 0.086100 \\
swallow_7B & 0.844000 & 0.079200 & 0.716000 & 0.123600 & 0.937200 & 0.015700 & 0.907200 & 0.037900 \\
youko_8B & 0.528000 & 0.429100 & 0.320000 & 0.340300 & 0.862400 & 0.109700 & 0.646000 & 0.218800 \\
\bottomrule
\end{tabular}
